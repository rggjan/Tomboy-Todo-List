\section{Requirements Overview}
\label{requirements}
We had to change our requirements document slightly to make it compatible with the end product. The revised requirements document (version 1.1) can be found in our downloads section at github\footnote{\url{http://github.com/downloads/rggjan/Tomboy-Todo-List/requirements-document-1.1.pdf}}.
In this section we refer to the 1.0 version of the requirements document\footnote{\url{http://github.com/downloads/rggjan/Tomboy-Todo-List/requirements-document-1.0.pdf}} and give a overview what we implemented (green), which requirements we implemented but in a slightly modified version (yellow) and which ones we did not implement (red).

\begin{tabular}{lll}
\rowcolor[gray]{0.9}
ID  & Requirement Title   & Detailed Explanation	\\
\completed	R.1		& Introducing Tasks   & 						\\
\completed	R.2		& Grouping Tasks together & 0   \\
\completed	R.3		& Priorities for Tasks & \\
\completed	R.4		& Due Dates for Tasks & \\
\completed	R.5		& Subtasks & \\
\notdone	R.6		& Exporting Tasks & Low Priority Requirement\\
\completed	R.7		& Create Task List & \\
\completed	R.8		& Edit Task List & \\
\completed	R.9		& Marking Tasks as Done & \\
\completed	R.10	& Automatically marking Tasks as Done & \\
\completed	R.11	& Setting a Due Date & \\
\parts		R.12	& Due Date visualization & \\
\completed	R.13	& Set Priority & \\
\completed	R.14	& Priority visualization & \\
\notdone	R.15	& Show / Hide completed Tasks & \\
\notdone	R.16	& Reorder Task Lists & \\
\completed	R.17	& Filter Notes & \\
\completed	R.18	& Tomboy directories & \\
\completed	R.19	& Tasks Persistence & \\
\notdone	R.20	& Tomboy Version & \\
\parts		R.21	& Platform independence & \\
\completed	R.22	& Default language & \\
\completed	R.23	& Translations & \\
\completed	R.24	& Documentation & \\
\completed	R.25	& Failures & \\
\completed	R.26	& Logging & \\
\completed	R.27	& Optimisation platform and measure & \\
\completed	R.28	& Response times & \\
\completed	R.29	& Initialising Tomboy & \\
\completed	R.30	& Git repository & \\
\completed	R.31	& Bug tracking & \\
\completed	R.32	& Source documentation & \\
\parts		R.33	& Unit testing & see \ref{testing}\\
\completed	R.34	& Installation from library & \\
\completed	R.35	& Activation of the addin & \\
\completed	R.36	& License compliance & \\
\completed	R.37	& Standard compliance & \\
\end{tabular}


\subsection{Testing}
\label{testing}
Unit testing proved to be a tricky part in our environment. We tried our best but we didn't manage to have complete coverage with our tests. But we think that we covered at least the most basic functionality for our addin (namely serialization/deserialization of tasklist).
The first problem was that our addin was written for Tomboy and that Tomboy itself was not written with unit tests in mind (Tomboy itself had only a small number (about 10) tests in their repository but most of them are outdated so they didn't run cleanly anymore). However since our addin on most parts only relied on the Note class of Tomboy we managed to instantiate those on our own and wrote some helper classes for that. With this we were able to write all of our tests.
Another issue is that a lot of our code is actually GUI related (like the tasklist behaviour) or at least very tightly coupled with the GUI and since we did not need to write tests for that we didn't do it, altough in our case it probably would have made sense.
In the end we had no tests for the subtasks/supertasks related code which was due to some bug which completely crashed monodevelop when we tried to run the tests for that. We think its because in the background Tomboy loads some code in the Note class when you have links in your text which does not work well when you intantiate note classes on your own, but we did not have the time anymore to investigate this more clearly.
We started writing tests after we already did 3 weeks of development. We think that was a too late and we certainly learned that if you want to do good testing you need to care about it from the start and always write your code having in mind that you need to be able to write tests for it. On the other hand it really took us some time to figure out how we can do unit tests in the first place since we had to find a way work around Tomboy.