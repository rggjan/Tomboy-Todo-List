\section{Individual Experiences}
\label{individual}

\subsection{Jan Rüegg}
\label{individual:jan}
For me, the project was a very good experience overall. I have to admit, though, that it made the java/C\# lecture \textit{by far} the most work-intensive lecture this Semester.

Still, the motivation in doing something like this is much higher than in other comparable projects that have a fixed task formulation or assignments. On one hand, because we could really do something on our own, something that we knew we could use later on, if we worked hard enough. But also because a contribution to an open source project like this is much more likely to eventually be used by other people, too, and maybe even integrated into a bigger project itself.

Despite all we put in there and the stress we had towards the end, our project is not really finished yet. But I think, it's at a point where one can start working with it, and where other developers, thanks to our wiki and documentation, can join and evolve it. And I, certainly, will keep working on the in my spare time, hoping that I and others find something useful in the result.

In the end, I think a good open source project is never really finished.

\subsection{Gabriel Walch}
\label{individual:gabriel}

This project left me with mixed feelings.
The challenging nature of it has really awaken my appetite for software engineering. However, the different problems that we had (see \ref{beginning}) soon began to tire me out.
Fortunately, communication inside the group went really well and we even managed to motivate each other by each pushing forward to achieve the goals we had set.
It was interesting for me (as I had not worked on anything similar before) to see how, in general, it could be kind of easy to extend existing open source projects with ones own ideas that one could really profit from it or even start from scratch and produce something usable.
On the other hand I realized in this project even more than in others that seemingly trivial things can turn out to be rather complex if set in the corresponding environment and that probably not as exciting tasks as documentation and maintenance (because they were missing in Tomboy, but helped us a lot in our own code for the others and ourselves) are a must have for efficient working.

\subsection{Gerd Zellweger}
\label{individual:gerd}
I think all in all the project was a good experience. Although it probably wasn't the most exiting code I wrote in my career as a student I liked that our work could be useful for a lot of people in the future. On the other hand I think it was good that we had to care about the stuff which normally doesn't matter in student projects (like documentation and testing). The project was time consuming but I think its OK since it also counts for the final grade of the course. In my opinion it could count even more than 50\% since its certainly more effort to finish the project than to learn for the exam.
I enjoyed working with my teammates and I think we worked well together. The communication and collaboration was very good. It think what helped here a lot was that we used great tools like git, github and Skype which simplified the teamwork a lot.