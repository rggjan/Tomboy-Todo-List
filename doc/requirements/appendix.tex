\clearpage
\appendix
\section{Appendix A: History of Task Management addins in Tomboy}

\label{appendix:history}
\subsection{Tasks}
There was a project (realized as an addin) for task management, called \textit{Tasks}, created by Boyd Timothy quite some time ago. However, this was removed in Version 0.9.3 of Tomboy, because there were too many problems with the existing implementation\footnote{\url{http://lists.beatniksoftware.com/pipermail/tomboy-list-beatniksoftware.com/2008-January/000540.html}}, i.e. bugs and missing integration into Tomboy.

Timothy abandoned the code and created a new program called \textit{Tasque}\footnote{\url{http://live.gnome.org/Tasque}} with its main focus just on task management.

\subsection{TaskLists}
In 2009 a new addin was initiated by Sandy Armstrong. This unfinished project, called \textit{TaskLists}\footnote{\url{http://gitorious.org/~sandy/tomboy/sandys-tomboy/commits/task-lists2}} is still in a very early development phase and has not much functionality yet. At the moment nobody is regularly working on it.

\subsection{Others}
Also, there are various inofficial addins that all have only a very small part of the functionality that we would like to implement. Examples are the \textit{Tomboy-Todo} addin \footnote{\url{http://romain.blogreen.org/Projects/Tomboy-Todo}} and the \textit{tasklist} addin\footnote{\url{http://www.ediweissmann.com/taskslist/}}.

\subsection{Conclusion}
\label{lessons}
One can say that the main problem (especially of the big \textit{Tasks} project) was, that real Tomboy integration was missing. That means, \textit{Tasks} had new windows for editing and managing the task lists, and was more like a separate program running \textit{besides} Tomboy, than a Tomboy addin. This is also the reason that a new, separate project \textit{Tasque} was created.

Also, problems were that projects projects were not really finished or were just too small to have functionality for \textit{everyday users}.

Our goal is to create an addin for task management that is big enough to be really useful, integrated enough into Tomboy to be easyly understood and used, and stable enough to be eventually integrated into Tomboy itself.
